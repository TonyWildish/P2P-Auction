\section{Discussion}

We have seen that the PSP auction can be used to calculate the allocation of network bandwidth shares in a multi-link network in a way that satisfies the requirements for LHCONE. This is clearly superior to having management-level meetings where people negotiate network shares for the next quarter, for example, but there remain a number of open questions.

The most important question is how to allocate budgets. In a true market setting the budgets are decided by the bidders themselves, the auction is then used to decide on the shares. In a setting such as LHCONE, where experiments or users are not paying with real money for the bandwidth they use, setting budgets becomes more problematic. Essentially, the budget has to represent something of real value to the bidder, so they consider it worth using rationally.

In the case of a single auction, for instance, if the budget is simply some form of fake money which has no value elsewhere, there is no incentive for the bidders to not spend their entire budget. This will lead to bidders bidding for more bandwidth than they actually need, with a consequent waste of resources. In the worst case, a bidder who has no need of bandwidth might bid on links that are valuable to a rival, essentially using their fake budget to launch a denial of service attack. Probably some mechanism is needed to penalize users who bid for resources that they do not use efficiently. If a user cannot make efficient use of their allocation they can simply trigger a new auction, and reduce their bids accordingly.

A related problem is that of multiple auctions, to handle renegotiation of allocations when a network-users' requirements change. Technically, there's no reason not to re-run the PSP auction whenever requirements change, or on some fixed schedule such as every hour. However, each auction will cost the successful bidders some of their fake budget, eventually leading to budget-depletion. This means budgets need to be replenished at intervals, but how this should happen in a world of fake money is not clear.

One possibility is to simply pay the bidders on a regular basis, according to some policy. However, this can lead to 'market distortions' if a bidder doesn't need their allocation for some period of time (perhaps due to experiment down-time). When the bidder re-enters the market they may have saved enough budget to dominate the bidding for some time, which can starve other bidders. IMposing a cap on the maximum budget may address this. In fact, this regard, allocating budgets bears a lot of resemblance to sharing CPU on a batch farm, and there are probably algorithms used there that can be adapted for this purpose.

Multiple auctions bring another consideration, that of learned strategies. In a single auction the bidders' best strategy is to tell the truth, but when auctions are repeated regularly it is possible that a bidder may learn the true valuation that other bidders place on a given resource. They may then adapt their strategies accordingly, to their advantage and to the detriment of other bidders. 

Collaboration is also possible, with users deciding to take turns to bid for a given resource. This is less problematic by definition, since it means that both parties are satisfied with the arrangement, but since it may still disadvantage other users of the network it needs to be considered.

Another open question is that of how to represent LHCONE. In principle the network can be represented literally, with each link in its proper place. Auctions can be held by each network provider for the links they maintain, with no need of coordination among different network providers. In practice, certain simplifications may be possible, based on how data actually flows through the network and on the actual needs of the users.

Finally, it is unreasonable to expect all network users to bid in an auction every time they interact with the network. There may be low-level users whose needs do not justify such complexity in their computing systems, and the overhead of managing many small circuits is not welcome. One simple solution is to have the network providers auction only a fraction of the available bandwidth, say maybe 80\%, leaving the remaining 20\% for other users. This will ensure that non-participants in the auction are not starved for resources by those who bid for bandwidth. However, it also means that if the reserved 20\% is not actually used at some point then the bidders who do want bandwidth may get less than they require unless they can overflow into the remaining bandwidth. A more effective, but more challenging solution, is to vary the reserved bandwidth fraction and trigger a new auction whenever non-participants present traffic to the network link. In this sense, soft bandwidth guarantees for circuits are more helpful than hard bandwidth guarantees, since they enable the flow of other traffic which can be monitored to detect a significant change in conditions.